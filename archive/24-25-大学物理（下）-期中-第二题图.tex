\documentclass{standalone}
\usepackage{preset}
\begin{document}
%24-25-大学物理(下)-期中-第二题图
\begin{tikzpicture}
	\begin{scope}
		\draw[draw=white,fill=white](-.2,-.95)rectangle(7.5,1.65);
		\draw(0,0)rectangle(4,-.4);
		\draw[rotate=10](0,0)rectangle(4,.4);
		\draw(0,0)--(0,-.8);
		\draw(4,0)--(4,-.8);
		\draw(3.6,0)rectangle(4.8,{3.6*tan(10)});
		\foreach \x in {0,.4,...,3.6}{
			\draw[<-](\x,{\x*tan(10)+.4/cos(10)})--(\x,{\x*tan(10)+.4/cos(10)+.5});
		}
		\draw[<->](0,-.7)--(4,-.7);
		\node[fill=white]at($(0,-.7)!.5!(4,-.7)$){\(L\)};
	\end{scope}
	\begin{scope}[shift={(6.2,.35)}]
		\draw(0,0)circle(1.2);
		\draw[color=gray,thick,domain=-1.16189:1.16189,smooth,variable=\y]plot({-.3-.4*exp(-8*\y*\y)},{\y});
		\draw[color=gray,thick,domain=-1.09088:1.09088,smooth,variable=\y]plot({.5-.4*exp(-8*\y*\y)},{\y});
		\path[<->]({-.3-.4*exp(-8)},-1)edge node[above]{\(b\)}({.5-.4*exp(-8)},-1);
		\draw(.1,0)--(.1,.8);
		\draw[->](-.1,.7)--(.1,.7);
		\path[->](.7,.7)edge node[below]{\(a\)}({.5-.4*exp(-8*.7*.7)},.7);
	\end{scope}
\end{tikzpicture}
\end{document}
